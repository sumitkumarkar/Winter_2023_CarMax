\documentclass[12 pt]{article}
\usepackage{geometry}
\geometry{a4paper, margin=1in}

\begin{document}
	

\begin{center}
	\textbf{\Large{Report of CarMax College Analytics Contest}\\\large{Name: Sumit Kumar Kar}}
\end{center}

\begin{enumerate}
\item \textbf{Price increases with Appraisal offer:} Price of a purchased car increases with the appraisal offer of an appraised car on an average.
\item \textbf{Market 14 shines!} The maximum percentage of transactions occur in Market 14 (9.86\%) and the minimum percentage of transactions occur in Market 13 (2.51\%). In fact, the number of cars purchased in any price range less than \$55k and the number of cars appraised less than \$20k are the maximum in Market 14. Furthermore, the combined transactions of only 5 (out 16) Markets is 41\%, namely, in Markets 4, 8, 10, 11, and 14. 
\item \textbf{In-person appraisals matter a lot:} Only 25\% appraisals are done online while 73\% appraisals are done in-person. In fact, the percentage of in-person appraisals is much more than the percentage of online appraisals in any market, any price range of appraised cars, and any price range of purchased cars. However, these percentages also vary significantly (statistically) with the market, the price range of appraised cars, and the price range of purchased cars.
\item \textbf{Noteworthy price range:} Roughly 70\% people purchase cars in the price range \$20k-\$35k while roughly 68\% people have appraisal offer in the range \$0-\$15k.
\item \textbf{Model year:} Model year of a person’s purchased car can be between 21 years older to 27 years younger than the appraised car. But 75\% of the people purchase cars at least 1 year younger than the appraised car and 50\% of the people purchase car at least 4 years younger than the appraised car.
\item \textbf{Nostalgia:} People seem to have a lot of attachment or likeness to some features of their appraised cars. 45\% people purchase a car with same body as the appraised car, 53\% people purchase the same number of cylinders, and 44\% people stick to the same trim description. In addition, 18\% people purchase a car with same make as the appraised car, (in fact 6.27\% people stick to the same model too), 19\% people purchase the same color, and 15\% people purchase the same engine type. Moreover, there is no (statistically) significant difference in the fuel capacity of the purchased car and appraised car of a person. However, there is a significant difference between the mileage (both city and highway) and horsepower.
\item \textbf{Surprising trim description:} Not only 44\% people stick to the same trim description, but also 20\% people trade in a non-premium car to purchase a premium car. But most surprisingly, 12\% people actually trade in a premium car to purchase a non-premium car. 
\item \textbf{Popular colors:} Be it a purchased or an appraised car, the most popular colors are white and black, followed by gray, and silver respectively. In fact, 76\% people purchase a car out of these colors and 70\% appraise a car out of these 4 colors.
\item \textbf{Cylinders matter:} Be it a purchased or an appraised car, the most popular number of cylinders is 4, followed by 6 and 8 respectively. In fact, 99\% of the (both purchased and appraised) cars have one of these 3 options and more than 55\% cars have 4 cylinders.
\end{enumerate}
	
	
\end{document}



